%!TEX TS-program = xelatex 
%!TEX TS-options = -synctex=1 -output-driver="xdvipdfmx -q -E"
%!TEX encoding = UTF-8 Unicode
%
%  monism_and_pluralism
%
%  Created by Mark Eli Kalderon on 2014-07-08.
%  Copyright (c) 2014. All rights reserved.
%

\documentclass[12pt]{article} 

% Definitions
\newcommand\mykeywords{color, pluralism}
\newcommand\myauthor{Mark Eli Kalderon}

% Packages
\usepackage{geometry} \geometry{a4paper} 
\usepackage{url}
\usepackage{txfonts}
\usepackage{color}
\usepackage{enumerate}
\definecolor{gray}{rgb}{0.459,0.438,0.471}
\usepackage{setspace}
% \doublespace % Uncomment for doublespacing if necessary
\usepackage{epigraph} % optional

% XeTeX
\usepackage[cm-default]{fontspec}
\usepackage{xltxtra,xunicode}
\defaultfontfeatures{Scale=MatchLowercase,Mapping=tex-text}
\setmainfont{Hoefler Text}

% Bibliography
\usepackage[round]{natbib}

% Title Information
\title{Monism and Pluralism}
\author{\myauthor} 
\date{} % Leave blank for no date, comment out for most recent date

% PDF Stuff
\usepackage[plainpages=false, pdfpagelabels, bookmarksnumbered, backref, pdftitle={Form Without Matter}, pagebackref, pdfauthor={\myauthor}, pdfkeywords={\mykeywords}, xetex, colorlinks=true, citecolor=gray, linkcolor=gray, urlcolor=gray]{hyperref} 

%%% BEGIN DOCUMENT
\begin{document}

% Title Page
\maketitle
% \begin{abstract} % optional
% \noindent
% \end{abstract} 
% \vskip 2em \hrule height 0.4pt \vskip 2em
\epigraph{Well, then, my friend, in the first place it is said that the earth, looked at from above, looks like those spherical balls made up of twelve pieces of leather; it is multi-colored, and of these colors those used by the painters give us an indication; up there the whole earth has these colors, but much brighter and purer than these; one part is sea-green and of marvelous beauty, another is golden, another is white, whiter than chalk or snow; the earth is composed also of the other colors, more numerous and beautiful than any we have seen.}{\textsc{Plato, \emph{Phaedo} 110 b--c}} % optional; make sure to uncomment \usepackage{epigraph}

% Layout Settings
\setlength{\parindent}{1em}

% Main Content

\section{Monism and Pluralism} % (fold)
\label{sec:monism_and_pluralism}

What is color pluralism? 

Not the claim that there are a plurality of colors, if such there be. Most philosophers are color monists, and if they are realists, they likely believe that there are a plurality of colors---that things are blue, yellow, and red, mauve and magenta, and many other colors, both named and unnamed. Nor is color pluralism the claim that objects can be multicolored. On the most straightforward understanding of that claim (not the only one), for an object to be multicolored is for it to have differently colored parts. But again the monist orthodoxy in the philosophy of color accepts that if things are colored they can be multicolored in the sense of having differently colored parts. To bring into focus the distinctive claim of color pluralism, it will be useful to contrast it with color monism. After all, color pluralism just is the denial of color monism. Though color monism is the orthodox position in the philosophy of color, it is rarely held explicitly with its commitments articulated clearly. So let us begin by examining the claims of color monism.

% So we shall begin by discussing color monism. 

There are a plurality of colors. Things are blue, yellow, and red, mauve and magenta, and many other colors both named and unnamed. But despite this plurality, all the colors that we see are, in some sense, generically alike. They are all colors. Specifically, they are sensible qualities of surfaces, transparent volumes, and radiant light sources that are perceptually available to sight (or would be if there were any) and that have a distinctive sensible character---a \emph{chromatic} character. What unites the colors that we see, if we do, as being the colors? (For an interesting skeptical case against the idea that the colors display the requisite unity see \citealt{Matthen:1999ma}.)

That there is a kind of unity manifest in a range of sensible qualities is an ancient idea. The ancients tended to think of a distinctive range of sensible qualities, such as color or temperature, as arrayed between opposites in order of their respective similarities to these (for discussion see \citealt{Lloyd:1966ly}). Thus there are a plurality of temperatures that we can feel. These are ordered between the extremes of hot and cold depending upon their similarities to these, in order of how hot or cold these temperatures are. Notoriously, Aristotle understands the colors on this model (\emph{De Sensu} \textsc{iii}). The colors are visible qualities arrayed between the extremes of light and dark ordered by their similarities to these. Thus Aristotle understood the hues to be a proportion of light and dark, a view with Homeric roots \citep{Gladstone:1858fk} that finds few modern defenders other than \citet{Goethe:1810uq}. Interestingly this Aristotelian doctrine persists in Walter Charleton's \citeyear{Charleton:1654fk} \emph{Physiologia}, a work overtly influenced by Gassendi's revival of Epicurean atomism and which shares Gassendi's aim of providing an alternative to Christianized Aristotelian natural philosophy. Given this, one might have expected a more Democritean view. (On Democritus see \citealt{Lee:2005qr,Pasnau:2007kx}; on the revival of Epicurean atomism in early modern philosophy see \citealt{Wilson:2008nr}). Even if, along with the early moderns, we reject the ancient view, we can hold onto the idea that the colors are generically alike. This need not be understood strictly in terms of the related notions of genus and species. Rather, the colors are generically alike, at least in part, by virtue of the relations they bear to one another. 

So far, we have the idea that there are distinctive ranges of sensible qualities, the colors prominently among them, that display some unity despite their plurality. Call such a distinctive range a \emph{family} of sensible qualities. The sensible qualities in a family display a unity in virtue of which they are generically alike. This unity is manifest in the relations the colors bear to one another. Following the ancients, in the case of the colors, this unity plausibly consists, at least in part, in the similarities and differences the colors bear to one another. We can accept at least that much, even if we reject the further ancient idea that the colors are arrayed in order of similarity to the extremes of light and dark. The colors are thus plausibly generically alike at least in the minimal sense of finding a place in a common color-similarity ordering. 

While we have extracted this much from the ancient view, we should be wary of looking no further. For there are other relations that obtain among the colors. Moreover, some of these obtain among the colors in virtue of what they are, in virtue of being the kind of sensible qualities they are (colors as opposed to temperatures, say) and having the specific sensible character that they do (a specific shade of mauve as opposed to a specific shade of magenta, say). Thus, as W.E. Johnson \citeyearpar{Johnson:1921fk} observed, colors stand in relations of determination, there are determinable and determinate colors \citep[for recent discussion see][]{Funkhouser:2006as}. Thus red is a determinable. There are different ways of being red, crimson and scarlet among them. Colors stand in relations of determination in virtue of what they are. Given the kind of sensible qualities they are, and given their specific sensible character, they stand in relations of determination. 

Moreover, not only are there similarities and differences among the colors as well as relations of determination, but the colors also stand in exclusion relations. Being red, a thing is not green. Being mauve, a thing is not magenta. And again this seems to be in virtue of what the colors are, the kind of sensible quality they are with their specific sensible character, visible chromatic qualities. 

So in addition to similarities and differences among the colors, perhaps what unites the plurality of sensible qualities in the chromatic family are relations of determination and exclusion. So far, then, we have considered three plausible candidates for relations obtaining among a plurality of sensible qualities in virtue of which they constitute a family of colors:
\begin{enumerate}
	\item similarities and differences
	\item determination relations between determinables and determinates
	\item exclusion relations
\end{enumerate}
For a relation to be a candidate for uniting the chromatic family, it must, at a minimum, hold of all and only the colors. Some say that red is like the sound of a trumpet. Perhaps so. Is this a problem for our first candidate relation? For it would seem that similarities obtain, not only among the colors, but among the colors and sounds. So to with differences. Aristotle in \emph{De Anima} \textsc{ii} maintained that we experience the difference between colors and sounds. In so doing he was self-consciously criticizing Plato who maintained, in the \emph{Theaetetus}, that discrimination was the operation of reason, not sense. Whether or not the difference between color and sound is apprehended by experience or reason, it remains the case that differences hold not only among the colors, but among the colors and other sensory objects as well. So similarities and differences don't hold among all and only the colors. That may be so, but what purports to unite, at least in part, the colors into a family of sensible qualities, are \emph{chromatic} similarities and differences. The similarity of red to the sound of a trumpet, and the difference between color and sound, are not chromatic similarities and differences. Similarly determination relations obtain among properties other than color qualities, but it is the structure of determinates and determinables in which they stand that purport to unite the colors into a chromatic family.

Once we have identified candidate relations that might serve to unite the colors into a chromatic family, we might ask not only whether they might in fact be so related but also how they might be. Perhaps the colors stand in some relations in virtue of standing in others. Even if all of some set of candidate relations genuinely obtain among all and only the colors, some candidate relations may be explanatorily prior to other candidate relations.

Thus, for example, the colors plausibly stand in exclusion relations as a consequence of their standing in a structure of determinates and determinables. Red is a determinable way for things to be in the sense that there is more than one way of being red. Scarlet is one way for a thing to be red, and crimson is another. So determinates are ways of being some determinable way. Moreover if something is scarlet it is not crimson and \emph{vice versa}. So distinct determinates of red, such as scarlet and crimson, are distinct ways of being that determinable way. This is why being scarlet excludes being crimson: the colors stand in exclusion relations as a consequence of their standing in a structure of determinates and determinables. It would seem then that determination relations are explanatorily prior to exclusion relations.

Similarities and differences on the one hand and the structure of determinates and determinables on the other are themselves importantly related, though it is a substantive and controversial issue which, if any, is prior to the other. Thus, for example, the structure of determinates and determinables can be represented by the geometry of the color space (see \citealt{Hilbert:2000on} and \citealt{Funkhouser:2006as}). Points in the color space represent utterly determinate colors (colors for which no other color is a determination), and determinable colors are represented by regions of the color space. Moreover, that a color is a determinate of a determinable, as scarlet is of red, is represented by the region of the color space associated with the determinate color being a subregion of the region associated with the determinable color. And two colors are codeterminates of a determinable if they are associated with nonoverlapping subregions of the broader determinable region. No doubt it is possible, at least in principle, to fully represent the structure of determinates and determinables among the colors in terms of the geometry of the color space. However, it is doubtful whether this fact, by itself, establishes the further claim that the colors stand in the structure of determinates an determinables \emph{because} of the similarities and differences between them. Consider the color space. Within it there is a dragon-shaped region coiling throughout that space without quite filling it. It is doubtful whether there is a color determinable corresponding to the dragon-shaped region. So it is not the case that for every region of the color space there is a color determinable that corresponds to it. So we cannot identify color determinables with arbitrary regions of the color space. If color determinables are regions of the color space, nonetheless, as they would be if color similarity and difference were explanatorily prior to the structure of determinates and determinables, they must be distinguished regions of that space. The challenge to the claim of explanatory priority is to specify these distinguished regions purely in terms of color similarities and differences. 

It is possible, at least in principle, to represent the structure of determinates and determinables among the colors in terms of the geometry of the color space. Another reason for doubting that this fact, by itself, suffices for the explanatory priority of color similarities and differences is that it is consistent, as well, with the reverse explanatory priority. Suppose that colors are similar or different from one another \emph{because} of the relations of determination in which they stand. It would be no surprise that determinates and determinables could be represented by the consequent relations of color similarity and difference that they give rise to. So a representation of the structure of determinates and determinables in terms of the geometry of the color space is consistent with the relations of determination having explanatory priority.

So far, we have the idea that there are distinctive ranges of sensible qualities, the colors prominently among them, that display some unity despite their plurality. Such a distinctive range is a \emph{family} of sensible qualities. The sensible qualities in a family display a unity in virtue of which they are generically alike. This unity is manifest in the relations the colors bear to one another. Plausibly, these include similarity and difference, determination, and exclusion relations, and perhaps others. Moreover, it is an open question what the precise explanatory relationship is between these relations. Despite these open questions, we should, by now, have a reasonably clear understanding of the notion of a family of colors.

Now that we have a clear if not distinct notion of a family of colors, we are in a position to define color monism:
\begin{quote}
	\emph{Color Monism}: There is one and only one family of colors
\end{quote}
Color monism presupposes realism about the colors. Consider color eliminativism, understood as the view that nothing is or could be colored. If color eliminativism is true, then there are no colors. And if there are no colors, there are no families of colors. And if there are no families of colors it is not the case that there is one and only one family of color qualities. So color monism presupposes realism about the colors. If we hold fixed this presupposed color realism, then color pluralism is the denial of color monism. Since we are presupposing color realism, we are presupposing that things are colored. Moreover, the colors of things are related to one another in such a way that colors are generically alike. The color pluralist concedes to the monist that there is a unity to a plurality of colors displayed in the relations of similarity and difference, determination, and exclusion in which they stand. And in virtue of standing in these relations these colors constitute a family of sensible qualities. The color pluralist merely denies that there is one and only one family of colors. Their pluralism consists in their conviction in there being a plurality of families of colors. According to color pluralism, then, things are multicolored, not merely in the straightforward sense of having differently colored parts, but in the sense of instantiating colors from different chromatic families (how this is so much as possible will become clear in sequel).

Why believe that there is a plurality of families of colors? Why depart from the monist orthodoxy? As we shall see in the next section, color pluralism is motivated as a response to the problem of conflicting appearances. The problem of conflicting appearances is a puzzle or \emph{aporia} at the heart of the Manifest Image of Nature. It is less a conflict between the Manifest Image of Nature and its Scientific Image (see \citealt{Sellars:1963eo}), than a conflict or incoherence within the Manifest Image of Nature itself. If the pluralist is right in contending that the \emph{aporia} is only resolved by denying color monism while retaining the monist's presupposed realism, then color monism is not only false but incoherent.

% section monism_and_pluralism (end)

\section{The Argument from Conflicting Appearances} % (fold)
\label{sec:the_argument_from_conflicting_appearances}

The label ``color pluralism'' was independently introduced by \citet{Mizrahi:2006zr} and \citet{Kalderon:2006tg}. The doctrine so labelled has antecedents, both ancient and more contemporary. Thus \citet{Kalderon:2006tg}, following \citet{Burnyeat:1979mv}, attributes this doctrine to Heraclitus. And, as we shall see, may contemporary philosophers are committed to color pluralism, even if they have never used that label (it is, at the very least, entertained by \citealt{Harman:2001mv}). The doctrine of color pluralism, that there is more than one family of colors, however it is labelled, is invariably introduced as a resolution of a puzzle or \emph{aporia}. Indeed the argument for color pluralism is a variant of the argument from conflicting appearances where the allegedly conflicting appearances are chromatic appearances. (On the argument from conflicting appearances see \citealt{Burnyeat:1979mv,Annas:1985fk}.)

A schematic representation of the puzzle is both useful and misleading. It is useful in that it allows one to clearly see the alleged inconsistency at the heart of the puzzle. However, it is in one important respect misleading. We will return to the way in which it is misleading when we are in a better position to appreciate this.

At the heart of the puzzle is three seemingly inconsistent claims that can be schematically represented as follows:
\begin{enumerate}
	\item \emph{Variation}: \( o \) appears \( F \) to \( S \) and \( o \) appears \( G \) to \( S' \)
	\item \emph{Incompatibility}: Nothing is both \( F \) and \( G \)
	\item \emph{Veridicality}: The \( F \)-appearance and the \( G \)-appearance are both veridical
\end{enumerate}
``\( o \)'' is a schematic letter whose permissible substituends are singular terms, be they names or definite descriptions, denoting colored objects. ``\( F \)'' and ``\( G \)'' are schematic letters whose permissible substituends are color predicates. Finally, ``\( S \)'' and ``\( S' \)'' are schematic letters whose permissible substituends are singular terms denoting subjects of perception. Following a taxonomy that goes at least as far back as the \emph{Theaetetus}, ``\( S \)'' and ``\( S' \)'' can be different perceivers from the same or different species, or they can be the same perceiver in different circumstances of perception.

Consider, then, the putative inconsistency. By \emph{Variation}, \( o \) appears \( F \) to \( S \) and \( o \) appears \( G \) to \( S' \). By \emph{Veridicality}, The \( F \)-appearance and the \( G \)-appearance are both veridical. So \( o \) is both \( F \) and \( G \). But that is straightforwardly inconsistent with \emph{Incompatibility}, the claim that nothing is both \( F \) and \( G \). 

It would seem, then, that in order to avoid this inconsistency, at least one of \emph{Variation}, \emph{Incompatibility}, or \emph{Veridicality} would have to be denied. Thus, color eliminativists avoid the puzzle by denying \emph{Veridicality} (see, for example, \citealt{Hardin:1993kn} on the location problem for the unique hues; for more recent discussion see \citealt{Gatzia:2010ga}). If nothing is or could be colored, then none of the color appearances that normal human perceivers enjoy are veridical. So there is no way for \emph{Variation} and \emph{Incompatibility} to conflict.

How does color pluralism avoid the conflict between \emph{Variation}, \emph{Incompatibility}, and \emph{Veridicality}? Unlike the color eliminativist, the pluralist retains the commitment to \emph{Veridicality}. The color pluralist thus retains the realism presupposed by the color monist. Instead, the pluralist denies \emph{Incompatibility}.

Before seeing how the denial of \emph{Incompatibility} leads to color pluralism, let's first consider how this is so much as possible. After all, \emph{Incompatibility} seems, at first blush, to be merely a consequence of exclusion relations obtaining among the colors, and we have conceded that exclusion relations are plausible candidates for being, in part, what unites the colors as a family of chromatic qualities. Does denying \emph{Incompatibility} commit one to denying that red excludes green, that being mauve excludes that thing from being at the same time magenta? (For an intriguing thought experiment that supports these latter possibilities see \citealt{Harman:2001mv}.) The pluralist maintains that one can deny \emph{Incompatibility} without denying that exclusion relations obtain among the colors or denying even that exclusion relations are plausibly among the relations that unite the colors into a family of chromatic qualities. Recall, the candidate relations that unite the sensible qualities into a family obtain among all and \emph{only} such qualities. Exclusion relations hold only within a family of colors. If mauve and magenta belong to the same chromatic family, then being mauve excludes being at the same time magenta. But this has no consequence for qualities outside that chromatic family. Something can be mauve and loud, say, or magenta and demure. Now suppose, the pluralist invites us to imagine, that there are plurality of families of colors. Since exclusion relations hold only within a family of colors, then while a determinate color from within a family will exclude all other determinate colors in that family, it will not exclude colors from distinct families, just as mauve does not exclude being loud, or magenta being demure.

According to the ancient taxonomy, \( S \) and \( S' \) can be different perceivers from the same or different species, or they can be the same perceiver in different circumstances of perception. Consider what is surely the strongest argument for color pluralism, the argument from conflicting appearances constructed on the basis of interspecies perceptual variation (see, \emph{inter alia}, \citealt{Bradley:2001mi,Byrne:2003we,Mizrahi:2006zr,Allen:2005be,Kalderon:2006tg}). Thus, plausibly, while humans perceive one family of colors, distinct species with color vision, such as bees and pigeons, perceive distinct families of colors. That these animals perceive distinct families of colors is made plausible by the fact that not only are distinct ranges of the electromagnetic spectrum visible (some birds and insects can see ultraviolet light invisible to normal human perceivers), but also by the fact that the colors perceived by different species can differ in dimensions of similarity. Thus pigeon colors, the colors perceptually available pigeons, have a dimension of color similarity that does not occur in the human color space \citep{Bradley:2001mi,Allen:2005be}. If humans and pigeons really do perceive distinct families of color qualities, then since exclusion relations hold only within families of colors, something can instantiate both a pigeon color, a color from the family of colors perceptually available to pigeons, and a human color, a color from a distinct family perceptually available to humans, all over and at the same time. Thus if the variation in appearance schematically represented by \emph{Variation} were the variation in color appearance between a human and a pigeon looking at the same scene, these appearances would not conflict, since being a pigeon color does not exclude being, at the same time, some human color, and so each appearance may be veridical.

The color pluralist while denying \emph{Incompatibility}, does not deny that colors from the same family exclude one another, nor that exclusion relations are among the relations that unite the colors into a family of chromatic qualities. The pluralist does not deny these things, since exclusion relations only obtain within a family, and the pluralist maintains that the variation in appearance is due to the visual presentation of colors from distinct families.

That is the abstract form of the pluralist response to the problem of conflicting appearances. Earlier I mentioned that the schematic representation of the conflict at the heart of the problem is importantly misleading. Now is a good time to explain why. There are a number of importantly different cases of perceptual variation that fit the abstract scheme. We have just discussed a case of interspecies perceptual variation, but there are also cases intraspecies perceptual variation, such as the variation in the location of the unique hues by normal human perceivers (see \citealt{Leon-M.-Hurvich:1968fu,Hardin:1993kn,Tye:2006lr,Cohen:2006fj,Tye:2006yq,Byrne:2007qy,Cohen:2007kx,Tye:2007fk,Kalderon:2006tg}; for arguments for color pluralism from intraspecies perceptual variation see \citealt{Byrne:1997dk,tye00,Mizrahi:2006zr,Kalderon:2006tg}). Whether the variation in appearance holds between members of the same or distinct species, each type of variation that we have so far considered is interpersonal. There is, according to the ancient taxonomy, in addition, cases of intrapersonal variation, where a scene appears differently to the same perceiver when viewed in different circumstances of perception. The way in which a colored object will appear differently to a perceiver in different conditions of illumination would be an example. The abstract representation of the problem of conflicting appearances obscures these important differences. This is important since different forms of perceptual variation might be differently explained, and this might have consequences for how best to resolve the seeming \emph{aporia}. 

Consider a case of perceptual variation which might give rise to the problem of conflicting appearances, but where the seeming paradox is implausibly resolved by making the pluralist response. Specifically, consider the variation in appearance involved in cases of color constancy (on the importance of constancy phenomena in the philosophy of perception see \citealt{smith02,Burge:2010uq}). Suppose you go out to the garden to pick a tomato to bring inside for washing and slicing. When you see the tomato outside in broad daylight, the tomato is seen to be a particular shade of red. As you move from natural daylight to the artificial illumination of the kitchen, the appearance of the red tomato changes without the tomato itself appearing to change color. As it travels between these differently illuminated environments, it appears to remain red, and that particular shade of red, even though it varies in appearance. This is a diachronic case of intrapersonal perceptual variation. There are synchronic cases as well (see \citealt{Cohen:2008hc}). Thus a partially shaded white while will appear uniformly white, though the shaded part of the wall appears differently from the brightly lit part of the wall. We might reasonably describe this difference in appearance by saying that while the brightly lit part of the wall appears white the shaded part appears gray. If we now reason that since nothing is both uniformly white and partly white and partly gray to the conclusion that the variable appearances in synchronic color constancy conflict with the appearance of a constant color, we have our puzzle or \emph{aporia}. The pluralist response to the problem of conflicting appearances gets going by claiming that the difference between the variable appearances is explained by their being the presentation of colors from different families. But not all variations in appearance are plausibly explained in terms of the presentation of different sensory objects. Perhaps the variation in appearance is best explained, not in terms of what is presented in appearance, but in terms of the way in which it is presented in appearance. Cases of color constancy are cases where what's presented is the constant color and what varies is the way that constant color is presented. This is Austin's \citeyearpar{Austin:1962lr} insight. Arguably, he inherits it from Aristotle (see \citealt{Kalderon:2015fr}). Austin's use of the Platonic example of the straight stick looking bent in a refracting medium (\emph{Republic} \textsc{x} 602c--603a) is precisely a case of shape constancy. And in maintaining that while the straight stick in water looks like a bent stick it does not look exactly like a bent stick, Austin is denying that the bent appearance is the presentation of anything bent. What is presented is the constant percept, what varies is the way it is presented. (See \citealt{Chisholm:1957dq,Kalderon:2010fj}. There are, of course, dissenters. For recent philosophers who claim that the variable appearances in cases of color constancy are best understood in terms of what is presented in those appearances see \citealt{Noe:2004fk,Chalmers:2006kx}).

However the variation in appearance involved in cases of color constancy are to be understood, the methodological point remains: Given that there are distinct sources of perceptual variation, susceptible to distinct explanations, it is implausible to suppose that every putative case of conflicting appearances admits of a uniform solution. It is therefore important to look at psychological and phenomenological details of the particular kind of perceptual variation before deciding which, if any, of \emph{Variation}, \emph{Incompatibility}, or \emph{Veridicality} to abandon. Insofar as the schematic representation of the problem of conflicting appearances obscures these differences thereby suggesting that all such cases should admit of a uniform solution, it is, to that extent, at least, misleading.

% section the_argument_from_conflicting_appearances (end)

\section{Metaphysical Accounting} % (fold)
\label{sec:metaphysical_accounting}

So far we have seen that color pluralism maintains that there are a plurality of families of colors, that things are multicolored, not merely in the sense of having differently colored parts, but in the sense that they can instantiate colors from different chromatic families all over at the same time. Thus something can have all over and at the same time a human color, a color perceptually available to humans, and a pigeon or a bee color, a color perceptually available to pigeons or bees. Moreover, color pluralism is argued to be the best resolution of certain, if not all, cases of the problem of conflicting appearances. Now that we have a better idea what color pluralism is and the reasons for it, let's clarify the metaphysical commitments of color pluralism.

First, let's revisit the pluralist's alleged commitment to color realism. The pluralist was represented as retaining the color monist's commitment to color realism. However, the attribution of realism to the color monist was made on the back of a particular, and particularly strong, characterization of color eliminativism---that nothing is \emph{or could be} colored. This suggests that the very existence of the colors is impossible and not merely their instantiation. The very existence of the colors would be impossible if the conditions for being color conflict, say, if they must at once be qualities of the natural environment obtaining independently of the perceiver and yet have the qualitative character only possessed by sensory experience (\citealt{Boghossian-Velleman:1989af,Boghossian-Velleman:1991as}). The denial of color realism can take weaker and stronger forms. If qualities need not be immanent but can be transcendent, if they can exist independently of their instantiation, then it is open to claim that while the colors exist, they are uniformly uninstantiated. This is a weaker form of denial than color eliminativism, at least as herein represented. This weaker denial of color realism, where the colors exist albeit uniformly uninstantiated, is consistent with the existence of a family of uninstantiated colors, qualities that, while uninstantiated, are importantly related by relations of similarity, difference, exclusion, and determination. Moreover, this weaker denial of color realism is thus also consistent with there being a plurality of families of uninstantiated colors. So perhaps color pluralism is consistent with the denial of realism after all. However, the principle reason for believing in color pluralism was in response to certain cases of conflicting appearances. Specifically, in certain cases, it is urged that the only way to resolve the \emph{aporia} within the Manifest Image of Nature is to deny \emph{Incompatibility} while retaining \emph{Variation} and \emph{Veridicality}. But notice, even the weaker form of the denial of color realism is committed to the denial of \emph{Veridcality}. That means that an irrealist color pluralism, where there are distinct families of uninstantiated colors, cannot be argued for on the basis of the argument from conflicting appearances. But if not for that reason, what reason could there be for holding this position? An irrealist color pluralism, while logically possible, seems unmotivated, at least for all that has been said (not necessarily all that can be said).

Some readers may have by now grown impatient. After all, it might be objected, color pluralists aren't the only ones who would resolve the problem of conflicting appearances by denying \emph{Incompatibility}. Protagorean relativists and relationalists more generally, deny incompatibility as well (for an important recent statement see \citealt{Cohen:2009lq}). Suppose that our colors, \( F \) and \( G \), are perceiver relative. To be sure being \( F \) for \( S \) will conflict with being \( G \) for \( S \). Being \( F \) for \( S \) will exclude being \( G \) for \( S \)---nothing is both \( F \) for \( S \) and \( G \) for \( S \) all over and at the same time. However, being \( F \) for \( S \) is perfectly compatible with being \( G \) for \( S' \). It is because the variable appearances involve the presentation of relational properties with different \emph{relata} that \emph{Incompatibility} fails. 

The objection, while \emph{prima facie} reasonable, is unfounded, however. Protagorean relativism about the colors is a species of color pluralism. According to the relativist, corresponding to each perceiver is a family of colors potentially determined by that perceiver in relation with the object and circumstances of perception. While exclusion relations hold within these families, relative colors from distinct families are compatible with one another. Notice that it is the claim that the perceived colors are from distinct families coupled with the claim that exclusion relations hold only within families of colors that resolves the paradox. Color relativism is color pluralism with an additional commitment to the colors in the plurality of families being relational in nature. Color pluralism, considered in and by itself, has no such commitment, though it is consistent with it. (We will discuss color relativism further in the next section.)

So far we have discussed color pluralism's relation to realism and relativism, but what of reductionism? One central question about the metaphysics of color is whether color qualities reduce to material or physical properties more generally. Reductionism is important since it offers the most straightforward answer to how the colors may be intelligibly realized by material surfaces, transparent volumes, and radiant light sources, thus reconciling the Manifest Image of Nature with its Scientific Image, at least with respect to our chromatic experience of the natural environment. Of course not all philosophers of color are reductionists. And not all that deny the possibility of reduction deny as well that colors are intelligibly realized in nature. Thus whereas \citet{Hilbert:1987jq,Byrne:1997dk,Byrne:2003we} maintain that families of colors are families of anthropocentrically defined physical properties, most likely reflectance types, color primitivists (such as \citealt{Campbell:1997dq,Broackes:1997pa,McGinn:1996oe,Yablo:1995fk,Gert:2008ge,Allen:2011bs}) deny the possibility of any such reduction but maintain that the colors may be intelligibly realized in nature (for a reductionist critique of primitivism, see \citealt{Byrne:2006bh}). Thus, for example, \citet{Yablo:1995fk} maintains that colors are nonphysical, but that they can be intelligibly realized by physical things since colors are nonphysical determinables with physical determinates (in contrast Byrne and Hilbert maintain that they are physical determinables with physical determinates). Color primitivism thus contrasts with eliminativist positions such as Hardin's \citeyearpar{Hardin:1993kn}. Like the primitivist, eliminativists deny the possibility of reduction, they differ only with respect to the consequences for the colors being intelligibly realized by the natural environment. Color pluralism, as characterized here, is neutral between reductionism and primitivism. The colors may be susceptible to physical reduction or not, but so long as there are a plurality of families of colors, then whether or not they are reducible to physical properties, color pluralism is true.

% section metaphysical_accounting (end)

\section{Objections and Replies} % (fold)
\label{sec:objections_and_replies}

In this final section, let's briefly consider the challenges and prospects of color pluralism as a response to certain cases of conflicting appearances.

Some objections to color pluralism apply to specific forms of color pluralism, others apply to all forms. Let's first consider an objection to a specific form of color pluralism, before considering more general objections.

Consider, then, color pluralism as a response to interpersonal variation in color appearances between normal human perceivers. Though a common scene viewed in the same circumstances of perception can present different chromatic appearances to different normal human perceivers, the pluralist maintains that these appearances do not conflict since these appearances differ only in the presentation of colors from distinct families and exclusion relations hold only within a family. One way, not the only way, of understanding this is that the visual sensibilities of the distinct perceivers select from among the plurality of abundant regularities that obtain in the natural environment different ranges of properties as being the colors. The color relativist, though no less a pluralist, does not accept the metaphor of selection and will understand the situation differently. Thus, according to the Protagorean relativist, colors are determined by the relation between the perceiver, the object of perception, and the circumstances of perception. So it is the presentation of different families of perceiver relative chromatic qualities that explains the difference in chromatic appearance of the common scene. Whether or not the pluralist response to interpersonal perceptual variation between normal human perceivers is best understood in relativist or non-relativist terms, both forms of pluralism face a common challenge. While the chromatic appearances enjoyed by normal human perceivers viewing a common scene in the same circumstances of perception may differ, if this difference is best explained by the presentation of colors from different chromatic families, how is a shared color language so much as possible? 

This is an ancient problem for Protagorean relativism. Socrates raises this objection in the \emph{Theaetetus} 183a--b (for discussion see \citealt{Burnyeat:1990dp}). \citet{Kalderon:2006tg} attempts to respond to this problem by emphasizing that the vast bulk of our color words are words that represent color determinables. Even very specific color words, such as ``burnt sienna'', represent colors that admit of further determinate shades. Moreover, it seems that the color words we have for determinate colors (leaving aside the artificial stipulations of philosophers in speaking of ``red\( _{17} \)'') are those that are definable in terms of color determinables. Thus ``unique green'' represents a determinate shade of green, but is definable as a shade of green that is not at all bluish and not all yellowish. But green, bluish, and yellowish are all determinables. The thought is that while we may not agree about determinate shades given the interpersonal variation in color appearance---we may disagree whether something is unique green, or bluish green, or yellowish green, nevertheless, we can agree that the perceived object is green .

% section objections_and_replies (end)

% Bibligography

\nocite{Cooper:1997fk}
\nocite{Hett:1936fk}
\nocite{Heraclitus:1979uq}

\bibliographystyle{plainnat} 
\bibliography{Philosophy} 

\end{document}
