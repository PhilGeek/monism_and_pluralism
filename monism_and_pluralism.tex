%!TEX TS-program = xelatex 
%!TEX TS-options = -synctex=1 -output-driver="xdvipdfmx -q -E"
%!TEX encoding = UTF-8 Unicode
%
%  monism_and_pluralism
%
%  Created by Mark Eli Kalderon on 2014-07-08.
%  Copyright (c) 2014. All rights reserved.
%

\documentclass[12pt]{article} 

% Definitions
\newcommand\mykeywords{color, pluralism}
\newcommand\myauthor{Mark Eli Kalderon}

% Packages
\usepackage{geometry} \geometry{a4paper} 
\usepackage{url}
\usepackage{txfonts}
\usepackage{color}
\usepackage{enumerate}
\definecolor{gray}{rgb}{0.459,0.438,0.471}
\usepackage{setspace}
% \doublespace % Uncomment for doublespacing if necessary
\usepackage{epigraph} % optional

% XeTeX
\usepackage[cm-default]{fontspec}
\usepackage{xltxtra,xunicode}
\defaultfontfeatures{Scale=MatchLowercase,Mapping=tex-text}
\setmainfont{Hoefler Text}

% Bibliography
\usepackage[round]{natbib}

% Title Information
\title{Monism and Pluralism}
\author{\myauthor} 
\date{} % Leave blank for no date, comment out for most recent date

% PDF Stuff
\usepackage[plainpages=false, pdfpagelabels, bookmarksnumbered, backref, pdftitle={Form Without Matter}, pagebackref, pdfauthor={\myauthor}, pdfkeywords={\mykeywords}, xetex, colorlinks=true, citecolor=gray, linkcolor=gray, urlcolor=gray]{hyperref} 

%%% BEGIN DOCUMENT
\begin{document}

% Title Page
\maketitle
% \begin{abstract} % optional
% \noindent
% \end{abstract} 
% \vskip 2em \hrule height 0.4pt \vskip 2em
\epigraph{Well, then, my friend, in the first place it is said that the earth, looked at from above, looks like those spherical balls made up of twelve pieces of leather; it is multi-colored, and of these colors those used by the painters give us an indication; up there the whole earth has these colors, but much brighter and purer than these; one part is sea-green and of marvelous beauty, another is golden, another is white, whiter than chalk or snow; the earth is composed also of the other colors, more numerous and beautiful than any we have seen.}{\textsc{Plato, \emph{Phaedo} 110 b--c}} % optional; make sure to uncomment \usepackage{epigraph}

% Layout Settings
\setlength{\parindent}{1em}

% Main Content

\section{Monism and Pluralism} % (fold)
\label{sec:monism_and_pluralism}

What is color pluralism? 

Not the claim that there are a plurality of colors, if such there be. Most philosophers are color monists, and if they are realists, all believe that there are a plurality of colors; that things are blue, yellow, and red, mauve and magenta, and many other colors, both named and unnamed. Nor is color pluralism the claim that objects can be multicolored. On the most straightforward understanding of that claim, for an object to be multicolored is for it to have differently colored parts. But again the monist orthodoxy in the philosophy of color accepts that if things are colored they can be multicolored in the sense of having differently colored parts. To bring into focus the distinctive claim of color pluralism, it will be useful to contrast it with color monism. After all, color pluralism just is the denial of color monism. Though color monism is the orthodox position in the philosophy of color, it is rarely held explicitly with its commitments articulated clearly. So let us begin by examining the claims of color monism.

% So we shall begin by discussing color monism. 

There are a plurality of colors. Things are blue, yellow, and red, mauve and magenta, and many other colors both named and unnamed. But despite this plurality, all the colors that we see are, in some sense, generically alike. They are all colors. Specifically, they are sensible qualities of surfaces, transparent volumes, and radiant light sources that are perceptually available to sight (or would be if there were any) that have a distinctive sensible character, a chromatic character. What unites the colors that we see, if we do, as being the colors?

That there is a kind of unity manifest in a range of sensible qualities is an ancient idea. The ancients tended to think of a distinctive range of sensible qualities, such as color or temperature, as arrayed between opposites in order of their respective similarities to these. Thus there are a plurality of temperatures that we can feel. These are ordered between the extremes of hot and cold depending upon their similarities to these, in order of how hot or cold these temperatures are. Notoriously, Aristotle understands the colors on this model (\emph{De Sensu} \textsc{iii}). The colors are visible qualities arrayed between the extremes of light and dark ordered by their similarities to these. Thus Aristotle understood the hues to be a proportion of light and dark, a view that finds few modern defenders other than \citet{Goethe:1810uq}. (Interestingly this Aristotelian doctrine persists in Walter Charleton's \citeyear{Charleton:1654fk} \emph{Physiologia}, a Restoration work overtly influenced by Gassendi's revival of Epicurean atomism and which shares Gassendi's aim of providing an alternative to Christianized Aristotelian natural philosophy. Given this, one might have expected a more Democritean view. On Decomcritus see \citealt{Lee:2005qr,Pasnau:2007kx}; on the revival of Epicurean atomism in early modern philosophy see \citealt{Wilson:2008nr}). Even if we reject the ancient view, we can hold onto the idea that the colors are generically alike. This need not be understood strictly in terms of the related notions of genus and species. Rather, the colors are generically alike, at least in part, by virtue of the relations they bear to one another. 

So far, we have the idea that there are distinctive ranges of sensible qualities, the colors prominently among them, that display some unity despite their plurality. Call such a distinctive range a \emph{family} of sensible qualities. The sensible qualities in a family display a unity in virtue of which they are generically alike. This unity is manifest in the relations the colors bear to one another. Following the ancients, in the case of the colors, this unity plausibly consists, at least in part, in the similarities and differences the colors bear to one another. We can accept at least that much, even if we reject the further ancient idea that the colors are arrayed in order of similarity to the extremes of light and dark. The colors are thus plausibly generically alike at least in the minimal sense of finding a place in a common color similarity ordering. 

While we have extracted this much from the ancient view, we should be wary of looking no further. For there are other relations that obtain among the colors. Moreover, some of these obtain among the colors in virtue of what they are, in virtue of being the kind of sensible qualities they are (colors as opposed to temperatures, say) and having the specific sensible character that they do (a specific shade of mauve as opposed to a specific shade of magenta, say). Thus, as W.E. Johnson \citeyearpar{Johnson:1921fk} observed, colors stand in relations of determination, there are determinable and determinate colors \citep[for recent discussion see][]{Funkhouser:2006as}. Thus red is a determinable. There are different ways of being red, crimson and scarlet among them. Colors stand in relations of determination in virtue of what they are. Given the kind of sensible qualities they are, and given their specific sensible character, they stand in relations of determination. 

Moreover, not only are there similarities and differences among the colors as well as relations of determination, the colors stand in exclusion relations. Being red, a thing is not green. Being mauve, a thing is not magenta. And again this seems to be in virtue of what the colors are, the kind of sensible quality they are with their specific sensible character, visible chromatic qualities. 

So in addition to similarities and differences among the colors, perhaps what unites the plurality of sensible qualities in the chromatic family are relations of determination and exclusion. So far, then, we have considered three plausible candidates for relations obtaining among a plurality of sensible qualities in virtue of which they constitute a family of colors:
\begin{enumerate}
	\item similarities and differences
	\item determination relations between determinables and determinates
	\item exclusion relations
\end{enumerate}
However, once we have identified candidate relations that might serve to unite the colors in a chromatic family, we might ask not only whether they might be so related but also how they might be. Perhaps the colors stand in some relations in virtue of standing in others. Even if all of some set of candidate relations genuinely obtain among all and only the colors, some candidate relations may be explanatorily prior to other candidate relations.

Thus, for example, the colors plausibly stand in exclusion relations as a consequence of their standing in a structure of determinates and determinables. Red is a determinable way for things to be in the sense that there is more than one way of being red. Scarlet is one way for a thing to be red, and crimson is another. So determinates are ways of being some determinable way. Moreover if something is scarlet it is not crimson and \emph{vice versa}. So distinct determinates of red, such as scarlet and crimson, are distinct ways of being that determinable way. This is why being scarlet excludes being crimson: the colors stand in exclusion relations as a consequence of their standing in a structure of determinates and determinables. It would seem then that determination relations are explanatorily prior to exclusion relations.

Similarities and differences on the one hand and the structure of determinates and determinables on the other are themselves importantly related, though it is a substantive and controversial issue which, if any, is prior to the other. Thus, for example, the structure of determinates and determinables can be represented by the geometry of the color space (see \citealt{Hilbert:2000on} and \citealt{Funkhouser:2006as}). Points in the color space represent utterly determinate colors (colors for which no other color is a determination), and determinable colors are represented by regions of the color space. Moreover, that a color is a determinate of a determinable, as scarlet is of red, is represented by the region of the color space associated with the determinate color being a subregion of the region associated with the determinable color. And two colors are codeterminates of a determinable if they are associated with nonoverlapping subregions of the broader determinable region. No doubt it is possible, at least in principle, to fully represent the structure of determinates and determinables among the colors in terms of the geometry of the color space. However, it is doubtful whether this fact, by itself, establishes the further claim that the colors stand in the structure of determinates an determinables \emph{because} of the similarities and differences between them. Consider the color space. Within it there is a dragon-shaped region coiling throughout that space without quite filling it. It is doubtful whether there is a color determinable corresponding to the dragon-shaped region. So it is not the case that for every region of the color space there is a color determinable that corresponds to it. So we cannot identify color determinables with arbitrary regions of the color space. If color determinables are regions of the color space, nonetheless, as they would be if color similarity and difference were explanatorily prior to the structure of determinates and determinables, they must be distinguished regions of that space. The challenge to the claim of explanatory priority is to specify these distinguished regions purely in terms of color similarities and differences. 

It is possible, at least in principle, to represent the structure of determinates and determinables among the colors in terms of the geometry of the color space. Another reason for doubting that this fact, by itself, suffices for the explanatory priority of color similarities and differences, is that it is consistent as well with the reverse explanatory priority. Suppose that colors are similar or different from one another \emph{because} of the relations of determination in which they stand. It would be no surprise that determinates and determinables could be represented by the consequent relations of color similarity and difference that they give rise to. So a representation of the structure of determinates and determinables in terms of the geometry of the color space is consistent with the relations of determination having explanatory priority.

So far, we have the idea that there are distinctive ranges of sensible qualities, the colors prominently among them, that display some unity despite their plurality. Call such a distinctive range a \emph{family} of sensible qualities. The sensible qualities in a family display a unity in virtue of which they are generically alike. This unity is manifest in the relations the colors bear to one another. Plausibly, these relations include similarities and differences, determination relations, and exclusion relations, and perhaps others. Moreover, it is an open question what the precise explanatory relationship is between these relations. Despite these open questions, we should, by now, have a reasonable understanding of the notion of a family of color properties.

Now that we are clearer on the notion of a family of color properties, we are in a position to define color monism:
\begin{quote}
	\emph{Color Monism}: There is one and only one family of color properties
\end{quote}
Color monism presupposes realism about the colors. Consider color eliminativism, the view that nothing is or could be colored. If color eliminativism is true, then there are no colors. And if there are no colors, there are no families of colors. And if there are no families of colors it is not the case that there is one and only one family of color properties. So color monism presupposes realism about the colors. If we hold fixed this presupposed color realism, then color pluralism is the denial of color monism. Since we are presupposing color realism, we are presupposing that things are colored. Moreover, the colors of things are related to one another in such a way that colors are generically alike. The color pluralist concedes to the monist that there is a unity to a plurality of colors displayed in the relations of similarity and difference, determination, and exclusion in which they stand. And in virtue of standing in these relations these colors constitute a family of sensible qualities. The color pluralist merely denies that there is one and only one family of color properties. Their pluralism consists in their conviction in there being a plurality of families of color properties.

Why believe that there is a plurality of families of color properties? Why depart from the monist orthodoxy? As we shall see in the next section, color pluralism is motivated as a response to the problem of conflicting appearances. The problem of conflicting appearances is a puzzle or \emph{aporia} at the heart of the Manifest Image of Nature. If the pluralist is right in contending that the \emph{aporia} is only resolved by denying color monism while retaining the monist's presupposed realism, then color monism is incoherent.

% section monism_and_pluralism (end)

\section{The Argument from Conflicting Appearances} % (fold)
\label{sec:the_argument_from_conflicting_appearances}

The label ``color pluralism'' was independently introduced by \citet{Mizrahi:2006zr} and \citet{Kalderon:2006tg}. The doctrine so labelled has antecedents, both ancient and more contemporary. The doctrine of color pluralism, that there is more than one family of color properties, however it is labelled, is invariably introduced as a resolution of a puzzle or \emph{aporia}. Indeed the argument for color pluralism is a variant of the argument from conflicting appearances where the allegedly conflicting appearances are chromatic appearances. 

A schematic representation of the puzzle is both useful and misleading. A schematic representation allows one to clearly see the alleged inconsistency at the heart of the puzzle. However, it is in one important respect misleading. We will return to the way in which it is misleading when we are in a better position to appreciate this.

At the heart of the puzzle is three seemingly inconsistent claims that can be schematically represented as follows:
\begin{enumerate}
	\item \emph{Variation}: \( o \) appears \( F \) to \( S \) and \( o \) appears \( G \) to \( S' \)
	\item \emph{Incompatibility}: Nothing is both \( F \) and \( G \)
	\item \emph{Veridicality}: The \( F \)-appearance and the \( G \)-appearance are both veridical
\end{enumerate}

% section the_argument_from_conflicting_appearances (end)

\section{Metaphysical Accoutning} % (fold)
\label{sec:metaphysical_accoutning}

% section metaphysical_accoutning (end)

\section{Objections and Replies} % (fold)
\label{sec:objections_and_replies}

% section objections_and_replies (end)

% Bibligography

\nocite{Cooper:1997fk}

\bibliographystyle{plainnat} 
\bibliography{Philosophy} 

\end{document}
