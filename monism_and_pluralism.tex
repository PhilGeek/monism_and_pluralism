Color Pluralism and Color Monism

What is color pluralism? Not the claim that there are a plurality of colors, if such there be. Most philosophers are color monists, and all believe that there are a plurality of colors, that things are blue, yellow, and red, mauve and magenta, and many other colors, both named and unnamed. Nor is color pluralism the claim that objects can be multicolored. On the most straightforward understanding of that claim, for an object to be multicolored is for it to have differently colored parts. But again the monist orthodoxy in the philosophy of color accepts that if things are colored they can be multicolored in the sense of having differently colored parts. To bring into focus the distinctive claim of color pluralism, it will be useful to contrast it with color monism. Color pluralism just is the denial of color monism. So we shall begin by discussing color monism. Though color monism is the orthodox position in the philosophy of color, it is rarely held explicitly with its commitments articulated clearly. So let us begin by examining the claims of color monism.

There are a plurality of colors. Things are blue, yellow, and red, mauve and magenta, and many other colors both named and unnamed. But despite this plurality, all the colors that we see are, in some sense, generically alike. They are all colors. Specifically, they are sensible qualities of surfaces, transparent volumes, and radiant light sources that are perceptually available to sight (or would be if there were any). What unites the colors that we see, if we do, as being the colors?

That there is a kind of unity manifest in a range of sensible qualities is an ancient idea. The ancients tended to think of a distinctive range of sensible qualities, such as color or temperature, as arrayed between opposites in order of their respective similarities to these. Thus there are a plurality of temperatures that we can feel. These are ordered between the extremes of hot and cold in order of their similarities to these, in order of how hot or cold these temperatures are. Notoriously, Aristotle understands the colors on this model. The colors are visible qualities arrayed between the extremes of light and dark ordered by their similarities to these. Thus Aristotle understood the hues to be a proportion of light and dark, a view that finds few modern defenders other than Goethe (though interestingly this Aristotelian doctrine persists in Walter Charlton's Physiologia, a work overtly influenced by Gassendi's revival of Epicurean atomism and which shares Gassendi's aim of providing an alternative to Christianized Aristotelian natural philosophy. Given this, one might have expected a more Democritean view). Even if we reject the ancient view, we can hold onto the idea that the colors are generically alike. This need not be understood strictly in terms of the related notions of genus and species. But, taking another idea from the ancients, the colors are generically alike, at least in part, by the similarities and differences between them.

So far, we have the idea that there are distinctive ranges of sensible qualities, the colors prominently among them, that display some unity despite their plurality. Call such a distinctive range a family of sensible qualities. The sensible qualities in a family display a unity in virtue of which they are generically alike. In the case of the colors, this unity consists, at least in part, in the similarities and differences the colors bear to one another. The colors are generically alike at least in the minimal sense of finding a place in a common color similarity ordering. 

While we have extracted this much from the ancient view, we should be wary of looking no further. For there are other relations that obtain among the colors. Moreover, some of these obtain among the colors in virtue of what they are, in virtue of being the kind of sensible qualities they are (colors as opposed to temperatures, say) and having the specific sensible character that they do (a specific shade of mauve as opposed to a specific shade of magenta, say). Thus, as W.E. Johnson observed, colors stand in relations of determination, there are determinable and determinate colors. Thus red is a determinable. There are different ways of being red, crimson and scarlet among them. Colors stand in relations of determination in virtue of what they are. Given the kind of sensible qualities they are, and given their specific sensible character, the stand in relations of determination. Moreover, not only are there similarities and differences among the colors as well as relations of determination, the colors stand in exclusion relations. Being red, a thing is not green. Being mauve, a thing is not magenta. And again this seems to be in virtue of what the colors are. 

So in addition to similarities and differences among the colors, perhaps what unites the plurality of sensible qualities in the chromatic family are relations of determination and exclusion. However, once we have identified a number of relations that might serve to unite the colors in a chromatic family, we might ask whether they might be related and how they might be. 




